\chapter{Why the reals?}

\section{Some intuitive statements}
Real numbers are the foundational block of real analysis and many of its characteristics and idiosinchrasies
come from the use of real numbers.
It is important to rembember that, even if they are an incredibly successful tool to modelize and predict 
reality, real numbers are a mental construct and are strongly associated to our everyday experience, our education
and the scale at which we experience the world.

In this chapter I will try to provide some arguments to support this statement. One of the main sources
that led me to this reasoning is \cite{Koerner2004}, a book I strongly recommend to the interested students.

We start by a statement, based on everyday experience and that would be considered trivially true in 
a conversation at the dinner table.

\begin{statement}
A bycicle travelling at constant velocity of $0.5$ meters per second in the direction of a wall,
at some point in time will hit the wall.
\end{statement}

We will use the language of mathematics to formalize this situation, as an experiment: we will put 
a long measurement tape on the ground, with the bycicle starting its run at the point $0$
and the wall being situated at the point corresponding to $50$ meters.

The movement of the front of the wheel of the bycicle can be described by the function 
$x(t) = 0.5 t$, and the statement that the bycicle is going to hit the wall is equivalent 
to the statement

\begin{statement}\label{st:eq}
Let $x(t) = 0.5 t$, then, there exists a $\tilde{t}$ such that $x(\tilde{t})=50$.  
\end{statement}

We can also think of this statement in a geometrical way: we imagine that the bycicle move along 
a line, and the wall is another line orthogonal to the first one, and the statement becomes

\begin{statement}\label{st:lines}
Two orthogonal lines cross at a point.
\end{statement}

\section{Questioning the statements}
An important epistemological, cultural and anthropogical question rises. 
The first statement depends on the way we look at the world and the scale at which we observe phenomena.
At the particle levels, phenomena as tunneling or the two slit experiment underline how our intuition 
and our formalization of linear movement are dependent on our everyday experience.

Statement \ref{st:eq} and \ref{st:lines} depend strongly on our costruction of the structure of the real numbers.
The two statements are equivalent to the following theorem.

\begin{theorem}
Let $f:[a, b] \subset \mathbb{R}\to \mathbb{R}$ be a continuous function such that 
$f(a)\leq 0 \leq f(b)$; then, there exists $c\in [a,b]$ such that $f(c)=0$.
\end{theorem}

Following \cite{Koerner2004}, we will show that this theorem is false if our domain are the rationals.
Since the rationals are a metric space, they are a topological space and we can define limits and 
continuity.
The following example, though, is a counterexample to the theorem above on the rationals.

\begin{example}
Let $f:\mathbb{Q}\to \mathbb{Q}$ such that 
\[
f(x)=\begin{cases}
    -1 & x< \sqrt{2}\\
    1 & x> \sqrt{2}
\end{cases}
\]
We restrict $f$ to $[0,2]$, then $f(0)<0$ and $f(2)>0$ but there is no $c$ such that $f(c)=0$ in $[0,2]$.
\end{example}

Or, by mirroring the statement in the section above; suppose our bycicle 
moves on the rationals, i.e., the position of our bycicle is described by 
$x(t) = 0.5t$ with $t\in \mathbb{Q}$; then our bycicle will never hit the wall at $x=\sqrt{2}$,
or, in other words, two orthogonal lines built on the rationals may not cross at one point.
We want the real numbers to mirror this intuition of the reality at our everyday scale, i.e.,
we want the real numbers to be complete.
In this course we will discuss the construction of the reals, starting from this idea of completeness.

What I would like to stress is that while the real numbers are an effective mean of description 
of reality, they are based on an intuition of the world and our vision of the world.
A huge part of contemporary mathematics is devoted to the study of real numbers, their good, 
their bad and their ugly.
It is important to stress that there are alternatives and other, better systems may be developed in the future,
but at the moment, the theoretical advantages and tradeoffs in other systems do not justify 
adopting them and real numbers are the de facto benchmark and is a fundamental part of 
the education of a mathematician to learn about them.

One of the main objectives of this course is to try to give a pragmatical justification on many
of the tools of real analysis, trying to explain why they matter in the wider panorama of science
and not as a collection of formal results.

\section{Identifying the main property}

\begin{definition}
A sequence in a set $X$ is a function that associates to each natural number 
an element of $X$.
\end{definition}


\begin{property}
Every increasing bounded sequence in $X$ has a limit.
\end{property}

\section{Construction of the reals: Dedekind cuts}